
%%%%%%%%%%%%%%%%%%%%%%% file typeinst.tex %%%%%%%%%%%%%%%%%%%%%%%%%
%
% This is the LaTeX source for the instructions to authors using
% the LaTeX document class 'llncs.cls' for contributions to
% the Lecture Notes in Computer Sciences series.
% http://www.springer.com/lncs       Springer Heidelberg 2006/05/04
%
% It may be used as a template for your own input - copy it
% to a new file with a new name and use it as the basis
% for your article.
%
% NB: the document class 'llncs' has its own and detailed documentation, see
% ftp://ftp.springer.de/data/pubftp/pub/tex/latex/llncs/latex2e/llncsdoc.pdf
%
%%%%%%%%%%%%%%%%%%%%%%%%%%%%%%%%%%%%%%%%%%%%%%%%%%%%%%%%%%%%%%%%%%%


\documentclass[runningheads,a4paper]{llncs}

\usepackage{amssymb}
\setcounter{tocdepth}{3}
\usepackage{graphicx}
\usepackage{float}
\usepackage[full]{complexity}
\usepackage{amsmath}
%\usepackage{amsfonts}
%\usepackage{amsthm}
\usepackage{subfigure}
%\usepackage{caption}
%\usepackage{subcaption}
%\usepackage{cite}
\usepackage{hyperref}
\usepackage{url}
\usepackage{clrscode4e}
\usepackage{verbatim}
\urlstyle{same}
\newcommand{\keywords}[1]{\par\addvspace\baselineskip
\noindent\keywordname\enspace\ignorespaces#1}

% Uniform numbering for previously defined theorem environments (e.g., LNCS).
\makeatletter
\let\c@lemma=\c@theorem
\let\c@corollary=\c@theorem
\let\c@fact=\c@theorem
\makeatother

% Redefinition of LNCS or SODA or Springer proof environment to put a \Box at
% the end of every proof.
\let\realendproof=\endproof
\def\endproof{\hspace*{\fill}$\Box$\realendproof}

\begin{document}

\mainmatter  % start of an individual contribution

% first the title is needed
\title{The Complexity of Determining Critical Sets}

% a short form should be given in case it is too long for the running head
\titlerunning{The Complexity of Determining Critical Sets}

% the name(s) of the author(s) follow(s) next
%
% NB: Chinese authors should write their first names(s) in front of
% their surnames. This ensures that the names appear correctly in
% the running heads and the author index.
%
\author{Fermi Ma \and Ariel Schvartzman \and Erik Waingarten}
%
\authorrunning{Fermi Ma \and Ariel Schvartzman \and Erik Waingarten}
% (feature abused for this document to repeat the title also on left hand pages)

% the affiliations are given next; don't give your e-mail address
% unless you accept that it will be published
\institute{MIT,\\
77 Mass Ave., Cambridge, MA 02139, USA, \\
\protect\url{{fermima,arielsc,eaw}@mit.edu}}

%
% NB: a more complex sample for affiliations and the mapping to the
% corresponding authors can be found in the file "llncs.dem"
% (search for the string "\mainmatter" where a contribution starts).
% "llncs.dem" accompanies the document class "llncs.cls".
%

\maketitle

\section{Upper Bounds}

Look at the decision problem of deciding whether there is a minimum clue of size less than $k$. Then
\begin{align}
(x, k) \in \mathsf{FCP} A \leftrightarrow \exists c_x \text{ of size $k$}, S_x \supset c_x, \forall S' \neq S_x, S' \supset c_x \text{ $S_x$ is a solution of $x$, and $S'$ is not.}
\end{align}
knowing the length of the minimum size clue with a binary search, we can find a minimum clue. So $\mathsf{FCP} A \in \Sigma_2^P$. 

\section{Lower Bounds}

Definitely $NP \subset \mathsf{FCP}$. Since we can solve $NP$-hard problems. We show that $\Sigma_2 \subset \mathsf{FCP}$. 

We take the following $\Sigma_2$ complete problem: $\exists x \forall y \phi(x,y) = 0$. We will write this as an equivalent $\mathsf{FCP}$ question, does there exists a setting of $k$ variables, such that for any setting of the rest, there is one unique solution?

The first thing we do is the following, suppose $|x| = k$
\begin{align}
\exists x \forall y \phi(x,y) = 0 &\leftrightarrow \exists 2k \text{ choices from $x, x', y$} \forall \text{rest} \phi(x,y) \vee \bigvee_{i=1}^k (x_i = x_i') = 0
\end{align}
Note that if we set the clue size to contain any $y$s, then there will be some $x$ or $x'$ not included, so we can always set $x_i = x_i'$ and satisfy the formula. Therefore, the $2k$ choices will be exactly those only specifying values of $x$ and $x'$. It remains to add a unique solution.
\begin{align}
& \leftrightarrow \exists 2k \text{ choices such that any of the rest $x, x', y, y^*$ } (\phi(x,y) \wedge (y^* = 0)) \vee (\bigwedge_i y_i=1) \vee  \bigvee_{i=1}^k (x_i = x_i') \\
	& \text{ has a unique solution}
\end{align}

Note that if $\phi(x,*)$ has no solutions, then our constructed formula has a unique solution after we pick $x, x'$. Still if $x, x'$ are not originally set, then picking one of the $x_i'$ or $x_i$ will give many solutions to the formula. If $\phi(x, *)$ has some solution for any $x$, then after we pick $x, x'$, then our constructed formula will have more than 1 solution. 

Therefore, $\Sigma_2 = FCP$. 
\end{document}
