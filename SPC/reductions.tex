
%%%%%%%%%%%%%%%%%%%%%%% file typeinst.tex %%%%%%%%%%%%%%%%%%%%%%%%%
%
% This is the LaTeX source for the instructions to authors using
% the LaTeX document class 'llncs.cls' for contributions to
% the Lecture Notes in Computer Sciences series.
% http://www.springer.com/lncs       Springer Heidelberg 2006/05/04
%
% It may be used as a template for your own input - copy it
% to a new file with a new name and use it as the basis
% for your article.
%
% NB: the document class 'llncs' has its own and detailed documentation, see
% ftp://ftp.springer.de/data/pubftp/pub/tex/latex/llncs/latex2e/llncsdoc.pdf
%
%%%%%%%%%%%%%%%%%%%%%%%%%%%%%%%%%%%%%%%%%%%%%%%%%%%%%%%%%%%%%%%%%%%


\documentclass[runningheads,a4paper]{llncs}

\usepackage{amssymb}
\setcounter{tocdepth}{3}
\usepackage{graphicx}
\usepackage{float}
\usepackage[full]{complexity}
\usepackage{amsmath}
%\usepackage{amsfonts}
%\usepackage{amsthm}
\usepackage{subfigure}
%\usepackage{caption}
%\usepackage{subcaption}
%\usepackage{cite}
\usepackage{hyperref}
\usepackage{url}
\usepackage{clrscode4e}
\usepackage{verbatim}
\urlstyle{same}
\newcommand{\keywords}[1]{\par\addvspace\baselineskip
\noindent\keywordname\enspace\ignorespaces#1}

% Uniform numbering for previously defined theorem environments (e.g., LNCS).
\makeatletter
\let\c@lemma=\c@theorem
\let\c@corollary=\c@theorem
\let\c@fact=\c@theorem
\makeatother

% Redefinition of LNCS or SODA or Springer proof environment to put a \Box at
% the end of every proof.
\let\realendproof=\endproof
\def\endproof{\hspace*{\fill}$\Box$\realendproof}

\begin{document}

\mainmatter  % start of an individual contribution

% first the title is needed
\title{The Complexity of Determining Critical Sets}

% a short form should be given in case it is too long for the running head
\titlerunning{The Complexity of Determining Critical Sets}

% the name(s) of the author(s) follow(s) next
%
% NB: Chinese authors should write their first names(s) in front of
% their surnames. This ensures that the names appear correctly in
% the running heads and the author index.
%
\author{Fermi Ma \and Ariel Schvartzman \and Erik Waingarten}
%
\authorrunning{Fermi Ma \and Ariel Schvartzman \and Erik Waingarten}
% (feature abused for this document to repeat the title also on left hand pages)

% the affiliations are given next; don't give your e-mail address
% unless you accept that it will be published
\institute{MIT,\\
77 Mass Ave., Cambridge, MA 02139, USA, \\
\protect\url{{fermima,arielsc,eaw}@mit.edu}}

%
% NB: a more complex sample for affiliations and the mapping to the
% corresponding authors can be found in the file "llncs.dem"
% (search for the string "\mainmatter" where a contribution starts).
% "llncs.dem" accompanies the document class "llncs.cls".
%

\maketitle

Given an $NP$ search problem, we have a set of instances $I$, and a set of solutions $S$. We let a clue be a subset of a solution.

The fewest clues problem asks for the smallest clue with a unique solution for a given instance.

\section{Reductions}

\subsubsection{General Clue Reduction}
We have a function mapping instances of $A$ to instances of $B$, $f: I_A \rightarrow I_B$ such that $x \in A \iff f(x) \in B$. We also want a function $g: C_B \rightarrow C_A$ such that $g$ is surjective, $g$ is a bijection on solutions, and subsets are preserved. So $c \subset c' \iff g(c) \subset g(c')$. We also require that if $c$ is a clue set for $f(x)$, then $g(c)$ is a clue set for $x$. 

So on input $x$, we find the minimum clue problem for $f(x)$, $c_{f(x)}$ and then use $g$ to get a clue $c_x$ for $x$.  

\begin{theorem}
Suppose $A$ reduces to $B$ with this reduction and $\mathsf{FCP} A$ is $\mathsf{FCP}$-hard, then $\mathsf{FCP} B$ is $\mathsf{FCP}$-hard. 
\end{theorem}

\begin{proof}
We want to show that if $c_{f(x)}$ is a minimum clue set, then $g(c_{f(x)})$ is a minimum clue set as well.

First, we show that $g(c_{f(x)})$ is a clue. This is true because $c_{f(x)}$ is a clue. That is, there exists some solution $S_{f(x)}$ such that $c_{f(x)} \subset S_{f(x)}$, which means that $g(c_{f(x)}) \subset g(S_{f(x)})$, which is a solution.

Second, we need to show that $g(c_{f(x)})$ has a unique superset solution. Suppose there were two different solutions, $S_x^{(1)}$ and $S_{x}^{(2)}$, then $g(c_{f(x}) \subset S_x^{(1)}, S_x^{(2)}$, and so $c_{f(x)} \subset S_{f(x)}^{(1)}, S_{f(x)}^{(2)}$ because $g$ is surjective, and because solutions have inverse solutions. Therfore, $c_{f(x)}$ is not unique.

Third, we need to show that $g(c_{f(x)})$ is minimal. Suppose there was some other $c' \subset g(c_{f(x)})$, since the $g$ is surjective, there exists some other $c_{f(x)}' \subset c_{f(x)}$ which is smaller. So by assumption, this must have two distinct solutions. But then mapping them again gives us two distinct solutions for $c'$. 

Therefore, $g(c_{f(x)})$ is a minimal clue set with a unique superset solution for $x$.
\end{proof}

Sometimes, the requirement that $g$ be a bijection on solutions is too restrictive, so we can give a more specialized sort of reduction that reduces the clue set.

\subsubsection{Special Subset Clue Reduction}
Suppose we have a problem $A$, with instance set $I_A$, solution set $S_A$, we let the set of clues be $C_A$. Note that $S_A \subset C_A$. We also have a problem $B$ with instance $I_B$, solution set $S_B$ and clue set $C_B$. 

We will first need a function $f: I_A \rightarrow I_B$ such that $x \in I_A$ has a solution if and only if $f(x) \in I_B$ has a solution. Note this is the normal concept of reduction.

We additionally specify $C_B' \subset C_B$ and an algorithm $M$ that maps any minimal clue set in $C_B$ into a minimal clue set in $C_B'$ of the same size which runs in polynomial time.

We will now let $g: C_B' \rightarrow C_A$ which satisfies the following properties:
\begin{enumerate}
\item $g$ is surjective.
\item $g$ is a bijection on solutions, and $S_x$ is a solution for $x$, if and only if $g^{-1}(S_x)$ is a solution to $f(x)$. 
\item $c \subset c' \iff g(c) \subset g(c')$. 
\end{enumerate}

\begin{theorem}
If we can find $f$, algorithm $M$, and function $g$ satisfying the above properties for $A$ and $B$, then if $\mathsf{FCP}A$ is $\mathsf{FCP}$-hard, then $\mathsf{FCP} B$ is $\mathsf{FCP}$-hard.
\end{theorem}

\begin{proof}
Suppose $x \in I_a$. We compute $y = f(x)$. If $x$ has no solution, we can identify it by noting that $y$ will have no solution.

Suppose $c_y$ is a fewest clues solution for $y$. Then $c_y' = M(c_y)$ will be another fewest clues solution to $y$. So for now on, suppose $c_y$ is a fewest clues solution for $y$ and $c_y \in C_B'$.

Then let $c_x = g(c_y)$. 

\begin{claim}
$c_x$ is a fewest clues solution to $x$.
\end{claim}

We know $c_y \subset S_y$ since $c_y$ is a clue. Which means that $g(c_y) \subset g(S_y)$.
Now we know that $c_x \subset g(S_y)$ and since $S_y$ is a clue of maximal size, then $c_x \subset S_x$, and by the property 2 of $g$, $S_x$ is a solution. So $c_x$ is a clue.

Suppose $c_x$ does not have a unique solution superset. So $c_x \subset S_x^{(1)}$ and $c_x \subset S_x^{(2)}$. Then since $g$ is bijective on solutions, let $S_y^{(1)}=g^{-1}(S_x^{(1)})$ and $S_y^{(2)}=g^{-1}(S_x^{(2)})$. So $S_y^{(1)}$ and $S_y^{(2)}$ are solutions to $y$, and so by property 3 of $g$, $c_y \subset S_y^{(1)}$ and $c_y \subset S_{y}^{(2)}$, which means $c_y$ does not have a unique superset solution. A contradiction.

Suppose $c_x$ is not minimal. Then let $c_x'$ be the smallest subset of $c_x$ with $S_x$ be the unique superset solution for $c_x'$. Then the smallest of $g^{-1}(c_x') = c_y'$ is a proper subset of $c_y$, and if there exists two superset solutions for $c_y'$, $S_y^{(1)}$ and $S_y^{(2)}$, then $c_x'$ does not have unique superset solutions.

This means that $c_x$ is the minimal set.
\end{proof}

\end{document}
