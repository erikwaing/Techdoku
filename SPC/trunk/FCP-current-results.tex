
%%%%%%%%%%%%%%%%%%%%%%% file typeinst.tex %%%%%%%%%%%%%%%%%%%%%%%%%
%
% This is the LaTeX source for the instructions to authors using
% the LaTeX document class 'llncs.cls' for contributions to
% the Lecture Notes in Computer Sciences series.
% http://www.springer.com/lncs       Springer Heidelberg 2006/05/04
%
% It may be used as a template for your own input - copy it
% to a new file with a new name and use it as the basis
% for your article.
%
% NB: the document class 'llncs' has its own and detailed documentation, see
% ftp://ftp.springer.de/data/pubftp/pub/tex/latex/llncs/latex2e/llncsdoc.pdf
%
%%%%%%%%%%%%%%%%%%%%%%%%%%%%%%%%%%%%%%%%%%%%%%%%%%%%%%%%%%%%%%%%%%%


\documentclass[runningheads,a4paper]{llncs}

\usepackage{amssymb}
\setcounter{tocdepth}{3}
\usepackage{graphicx}
\usepackage{float}
\usepackage[full]{complexity}
\usepackage{amsmath}
%\usepackage{amsfonts}
%\usepackage{amsthm}
\usepackage{subfigure}
%\usepackage{caption}
%\usepackage{subcaption}
%\usepackage{cite}
\usepackage{hyperref}
\usepackage{url}
\usepackage{clrscode4e}
\usepackage{verbatim}
\urlstyle{same}
\newcommand{\keywords}[1]{\par\addvspace\baselineskip
\noindent\keywordname\enspace\ignorespaces#1}

% Uniform numbering for previously defined theorem environments (e.g., LNCS).
\makeatletter
\let\c@lemma=\c@theorem
\let\c@corollary=\c@theorem
\let\c@fact=\c@theorem
\makeatother

% Redefinition of LNCS or SODA or Springer proof environment to put a \Box at
% the end of every proof.
\let\realendproof=\endproof
\def\endproof{\hspace*{\fill}$\Box$\realendproof}

\begin{document}

\mainmatter  % start of an individual contribution

% first the title is needed
\title{The Fewest Clues Problem}

% a short form should be given in case it is too long for the running head
\titlerunning{The Fewest Clues Problem}

% the name(s) of the author(s) follow(s) next
%
% NB: Chinese authors should write their first names(s) in front of
% their surnames. This ensures that the names appear correctly in
% the running heads and the author index.
%
\author{Fermi Ma \and Ariel Schvartzman \and Erik Waingarten}
%
\authorrunning{Fermi Ma \and Ariel Schvartzman \and Erik Waingarten}
% (feature abused for this document to repeat the title also on left hand pages)

% the affiliations are given next; don't give your e-mail address
% unless you accept that it will be published
\institute{MIT,\\
77 Mass Ave., Cambridge, MA 02139, USA, \\
\protect\url{{fermima,arielsc,eaw}@mit.edu}}

%
% NB: a more complex sample for affiliations and the mapping to the
% corresponding authors can be found in the file "llncs.dem"
% (search for the string "\mainmatter" where a contribution starts).
% "llncs.dem" accompanies the document class "llncs.cls".
%

\maketitle

\section{Introduction}
\label{sec:introduction}

\section{Related Work}
\label{sec:relatedwork}

\section{The FCP Complexity Class}
\label{sec:complexityclass}

\subsection{$FCP-SAT$}

\section{$\mathsf{FCP}$-Complete Problems}

\subsection{Reductions}

\subsection{3SAT}

\subsection{TrianglePartition}

\subsection{Latin Squares}

\subsection{Sudoku}

\section{$FCP = \Sigma_2$}

\subsection{$FCP \subseteq \Sigma_2$}

\subsection{$FCP \supseteq \Sigma_2$}

\section{$\mathsf{FCP}$ Versions of Easy Problems}

\subsection{FCP 2SAT} 
(Fermi)

\subsection{Other P problems with NP-hard FCP versions}

(open)

\subsection{P problems with $\Sigma_2$-hard FCP versions}

(open)

\section{Conclusion}
\label{sec:conclusion}


\bibliography{references}
\bibliographystyle{splncs}
\end{document}
