\documentclass[11pt]{article}
\usepackage{graphicx}    % needed for including graphics e.g. EPS, PS
\topmargin -1.5cm        % read Lamport p.163
\oddsidemargin -0.04cm   % read Lamport p.163
\evensidemargin -0.04cm  % same as oddsidemargin but for left-hand pages
\textwidth 16.59cm
\textheight 21.94cm 
%\pagestyle{empty}       % Uncomment if don't want page numbers
\parskip 7.2pt           % sets spacing between paragraphs
%\renewcommand{\baselinestretch}{1.5} % Uncomment for 1.5 spacing between lines
\usepackage{amsmath}
\usepackage{amsfonts}
\usepackage{verbatim}
\usepackage{blkarray}
\parindent 0pt		 % sets leading space for paragraphs
\author{Fermi Ma}
\title{The Complexity of Completing Partial Latin Squares}
\newcommand{\x}{\overline{x}}
\newcommand{\fc}{\frac{x_C}{N}}

\begin{document}
\maketitle

\section{Overview}

An order $n$ Latin square is an $n \times n$ array where each entry is an element from the set $\{1,\dots,n\}$. Each row and column ontains each element exactly once. The problem of completing a partial Latin square is NP-complete. This is proven in two steps. First, the authors show a correspondnce between the problem of completing Latin squares and the graph-theoretical problem of partitioning tripartite graph into triangles, which they show is NP-complete. After that, the authors use techniques by Hall [citation 5 in this pper] and Ryser [citation 13] to reduce triangle-partition of tripartite graphs to completion.

\section{The Defect Graph}

Given a partial Latin square $P$ of order $n$, a \emph{defect} of the latin square is the graph $G(P) = (V,E)$ where
\begin{align*}
V = &\{r_i | \text{ row }i\text{ contains an empty square}\} \\
&\cup \{c_j | \text{ column }j\text{ contains an empty square}\} \\
&\cup \{e_k | \text{ element }k\text{ appears in fewer than }n\text{ squares}\}
\end{align*}
and 
\begin{align*}
E = &\{(r_i,c_j) | \text{ the }(i,j)\text{ square of }P\text{ is empty}\} \\
&\cup \{(r_i,e_k) | \text{ row }i\text{ does not contain element }k\} \\
&\cup \{(c_j.e_k) | \text{ column }j\text{ does not contain }k\}
\end{align*}

This defect graph is a tripartite graph that has a triangle-partition if and only if $P$ is a Latin square that can be completed.

\section{Latin Frameworks}

In this section, we introduce the idea of a Latin framework for a tripartite graph. The intuition is as follows: under certain constraints, the Latin framework of a graph is a partial Latin square such that the original graph is the defect of this Latin square.

Given a tripartite graph $G = (V,E)$ with tripartition $V_1 \cup V_2 \cup V_3$, we label the vertices in $V_1$ as $r_1,r_2,\dots,r_{|V_1|}$, the vertices in $V_2$ as $c_1,c_2,\dots,c_{|V_2|}$, and the vertices in $V_3$ as $e_1,e_2,\dots,e_{|V_3|}$. It does not matter which vertex in each $V_i$ gets which label.

A Latin framework for such a $G = (V_1 \cup V_2 \cup V_3,E)$, denoted $LF(G;r,s,t)$ is an $r \times s$ array where each entry is either empty, or in the set $\{1,2,\dots,t\}$. Each row and column contains each element at most once, and the following additional constraints must be satisfied by $LF(G;r,s,t)$:
\begin{enumerate}
\item If $G$ contains the edge $(r_i,c_j)$, the $(i,j)$ entry is empty. If the edge is not present, the $(i,j)$ entry must be filled with an element from $\{1,2,\dots,t\}$
\item If $G$ contains the edge $(r_i,e_k)$, row $i$ does not contain $k$.
\item If $G$ contains the edge $(c_j,e_k)$, column $j$ does not contain $k$
\end{enumerate}

Observe that when $r = s = t$, $G$ is the defect of $LF(G;r,r,r)$. 

We now give the following results.

\textbf{Lemma 1:} For $G = (V_1 \cup V_2 \cup V_3,E)$, there exists a Latin framework $LF(G;n,n,2n)$ where $n = |V_1| + |V_2| + |V_3|$.

\emph{Proof:} We let $LF(G;n,n,2n)$ be the $n \times n$ array where the $(i,j)$ entry is empty if $(r_i,c_j \in E)$, and contains $1 + n + ((i+j) \mod n)$ if $(r_i,c_j \not\in E)$.

\textbf{Lemma 2:} Given a Latin framework $LF(G;n,k,2n)$ where $n \leq k \leq 2n$, we can extend this to a Latin framework $LF(G;n,k+1,2n)$.

\emph{Proof:} See appendix

\textbf{Theorem 1:} For $G = (V_1 \cup V_2 \cup V_3,E)$, there exists a Latin framework $LF(G;2n,2n,2n)$ where $n = |V_1| + |V_2| + |V_3|$.

\emph{Proof:} We use Lemma 1 to write down $LF(G;n,n,2n)$. We use a repeated application of Lemma 2 to add $n$ columns to get $LF(G;n,2n,2n)$, and then we transpose the machinery from Lemma 2 to add $n$ rows to get $LF(G;2n,2n,2n)$.

\section{NP-completeness of Completing Partial Latin Squares}

We consider the problem of finding triangle partitions in a tripartite graph. First, we make the following observation about this problem. Call a tripartite graph uniform if every node has an equal number of neighbors in each of the other two partitions. A tripartite graph does not have a triangle partition if it is not uniform. 

\textbf{Theorem 2:} Deciding whether a tripartite graph has a triangle partition is NP-complete.

\emph{Proof:} See appendix.

Now we are ready to prove the main result.

\textbf{Theorem 3:} Deciding whether a partial Latin square can be completed is NP-complete.

\emph{Proof:} Membership in NP follows from the fact that the completed Latin square can be used as a certificate that a partial Latin square can be completed.

It remains to show NP-hardness. We do this via reduction from the triangle partition problem. Given a tripartite graph $G$, we first check whether or not it is uniform. If it is not uniform, then there does not exist a triangle partition. If it is uniform, we use Theorem 3 from the Latin frameworks section to write down a latin framework $LF(G;2n,2n,2n)$ in polynomial time. Since the Latin framework is constructed so that $G$ is a defect of $LF(G;2n,2n,2n)$, there is a triangle partition of $G$ if and only if we can complete the partial latin square $LF(G;2n,2n,2n)$. This completes the proof.

\section{Appendix}

Here, we prove the claim that deciding whether a tripartite graph has a triangle partition is NP-complete.

\emph{Proof: } Membership in NP follows from the fact that the triangle partition itself can be used as a certificate that constructing a partition is possible.

It remains to show NP-hardness. FINISH THIS SECTION.

Lemma 2.
\end{document}
